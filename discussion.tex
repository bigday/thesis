\chapter{Discussion}
% TODO: reflect on the work done; analyze the choices made etc. as objectively
% as possible

In this chapter, we discuss the work completed towards reconstructing a 3D mesh of the human body and outline suggestions for making a comprehensive implementation. We also analyze the choices made and the operability of different parts of the system. Finally, we conclude by reviewing the major achievements of this thesis, while examining problems in planning and conducting the research.

\fixme{ \\
    TODO: review the best/chosen approach step by step \\
    - kinect data via openni \\
    - user segmentation by NITE \\
    - body part segmentation (nearest neighbor bone) \\
    - automatic measurements on bones and body parts \\
    - ??? \\
    - makehuman measurement plugin \\
    - (blender for animation) \\
}


\section{Future work}

\section{Conclusion}

The major accomplishments of this thesis are threefold in regard to human body reconstruction using depth camera data.

Firstly, we conducted a review of the subject matter, including existing systems and related research. In chapter~\ref{literature}, we described the most relevant areas of research in terms of what should be considered when designing a system for 3D reconstruction of the human body. We also encountered and identified the major problems that must be solved in order to create a working system.

Secondly, we suggested different valid approaches in chapter~\ref{approach}. We described each approach in terms of how the problem can be divided into sub-tasks and determined requirements for their implentations. We also analyzed possible problems with each approach and recommended the one we deemed most promising.

Thirdly, we made prototype implementations for many of the tasks described, as described in chapter~\ref{approach}. These implementations helped us further understand the tasks and problems that must be solved. They also provide a useful basis for further work on a complete human body reconstruction system.

\newtopic

Ultimately, this research did not meet its original goal---a complete human body reconstruction system that works in real time. Various causes contributed to this result. Fundamentally, the difficulty of achieving the chosen research goal was seriously underestimated. Our lack of previous knowledge of the subject matter was the primary reason for this. Unfortunately, there was little existing expertise in relevant research topics in the Department of Media Technology, and instruction during the research was lacking. Therefore, it took us long to realize our mistake.

Still, we have made a promising start in the research---the error was mostly in choosing the scope to fit the limited resources of a Master's thesis. Scoping the research appropriately was made difficult by the broadness of the subject. Any given task requires a broad understanding of different research areas, which had to be gained first. We also made the error of focusing on finding a lot of different ideas---while neglecting validation of their feasibility before committing to working on them.

All things considered, we made good groundwork by investigating, planning and partially implementing a system for human body reconstruction, while the research goal was not completely met.
