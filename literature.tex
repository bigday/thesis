\chapter{Related work}

\fixme{Refer to relevant sources a topic at a time}

% libraries
% \citep{OpenNI} \citep{NITE} \citep{PCL}
\section{Point cloud alignment} \label{literature.alignment}

\fixme{Why is point cloud alignment interesting?}

The Iterative Closest Point (ICP) algorithm is commonly used for aligning two point clouds. \fixme{The algorithm was originally presented where, by whom?} The algorithm in pseudocode is shown in \ref{literature.pseudoicp}. It works by finding corresponding points from the two point clouds. Then a rigid transformation that minimizes the error between the corresponding point pairs is searched for. This transformation is then applied. Iterating this multiple times, a local optimum should be found.

\fixme{make this a float or something}
\begin{algorithm}
\label{literature.pseudoicp}
for 1..iterations:
  for each point $x_i$ in X:
    $y_i$ = closest point in Y
  error := $sum_i (x_i-y_i)^2$
  find rigid transformation $T_i$ that minimizes error
  apply $T_i$ on X
output T = $product_i T_i$
\end{algorithm}

Even from this description it's obvious that ICP has its shortcomings. If the initial guess is too much off, the algorithm can yield horrible results. This is because for non-trivial point clouds, most local optima are significantly different from the global optimum.

Another issue with ICP is that it requires a lot of computation if the point clouds are large. \citep{rusinkiewicz2001efficient} review different ways of improving the efficiency of ICP. They also attain minor improvements to the alignment accuracy.

However, even these improved variants of ICP suffer from converging on local minima. The EM-ICP \fixme{(abbreviation)} algorithm introduced by \citep{granger2006multi} overcomes this problem at the cost of being computationally very intensive. 

\citep{tamaki2010softassign} make EM-ICP more efficient by implementing the computation on GPU using CUDA. The speed is unfortunately still not fast enough for real-time usage on current hardware.

\fixme{Another interesting improvement over ICP is suggested by \citep{tykkala2011direct}. Their algorithm, Direct Iterative Closest Point, is very well suited for SLAM using an RGB-D sensor. (try to understand this better and write more accurately)}

\citep{huang2011visual}


\section{Human body models}
\citep{anguelov2005scape}
\citep{baek2012parametric}

\section{3D reconstruction}
(generic)
\citep{fabio2003point}

(kinect fusion)
\citep{izadi2011kinectfusion}
\citep{newcombe2011kinectfusion}
\citep{Whelan12rssw}

(human mesh)

Currently, the most promising work at creating personalized human avatars has been by \citet{weiss2011home}. Their approach uses the SCAPE body model \citep{anguelov2005scape}. This makes for very good reconstruction accuracy and excellent \fixme{rendered shapes (muscle deformation according to pose etc)}. However, the evaluation is slow\fixme{ at x seconds per frame (or somesuch)} and the SCAPE model needs to be trained with a large amount of data beforehand.

\citep{schneider2010fitting}
\citep{ahmed2005automatic}
\citep{tongscanning}
\citep{charpentier2011accurate}
\citep{hirshbergc2011evaluating}

(human pose)
\citep{baak2011data}
