\chapter{Introduction}

\fixme{Background:} Depth cameras have recently become cheap and widely available. While the technology is old, there were no consumer devices before Microsoft introduced their Kinect for Xbox 360. Kinect contains a normal camera for visible light, \fixme{and} also includes an IR camera and a laser emitter for depth measurements. \fixme{TODO: write about technologies -- time-of-flight, triangulation etc. (perhaps in literature section?)}

Originally, Kinect was only supported for Xbox 360. However, hobbyists quickly began projects to connect the Kinect to a PC. The API was soon reverse engineered. Later on Microsoft released an SDK to the public, allowing people to experiment with Kinect on a Windows PC.

Afterwards, other consumer sensors were released. \fixme{Asus Xtion Pro/Live, what else}

\fixme{TODO: explain the problem we're trying to solve here}
Cheap depth cameras can already be used for 3D scanning static objects at home. However, to date there has been no software available for scanning humans moving freely. There's no fundamental reason why this can't be done. Perhaps the most interesting moving thing that people would like to scan is people. The possibility to create a 3D model of oneself is intriguing. It also would have an enormous amount of applications, same as ordinary photography.

\fixme{TODO: make explicit constraints, i.e. formulate the problem accurately. Present a research question.}

The possibilities granted by a technology to efficiently scan moving people are vast, but so are the technological details that could be accounted for. In an effort to pick a reasonable research subject, some seemingly arbitrary constraints must be chosen.

Firstly, we choose to pursue an actual working implementation (no matter how crude) instead of only a description of methods that have perhaps been successfully applied manually. We try to reach an implementation that doesn't require anything extraordinary to run. Therefore we require that the implementation works on a high-end PC using one Microsoft Kinect. We arbitrarily choose to develop on Linux, but strive to make the implementation multiplatform.

Finally, we require that the implementation works in real time. This is because good methods for offline modeling have already been presented (though there's no implementation available).

The previously mentioned problem and constraints are formulated as the following research question:

\fixme{Is it possible to generate a 3D model of the user in real time, using Microsoft Kinect and a consumer PC?}
