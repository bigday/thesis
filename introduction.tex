\chapter{Introduction}

Humans have always been interested in other humans. Plenty of sculptures and cave paintings depicting humans have been found that are thousands or even tens of thousands of years old. As time went on, new methods of visually recreating the human form were found---paintings on canvas, reliefs, life-size statues. Different kinds of works were created in different periods, while humans always remained an important subject.

As technology advanced, creating visual replications of humans became notably easier. With the advent of cameras, the skill of drawing or painting could be replaced with the more technical skills of using the camera and developing the film. The popularization of cameras has led to photography itself becoming an art form, while the technology itself has become very easy to use.

The age of computers has come since. The progress of computer graphics has inevitably lead to attempts at creating three-dimensional (3D) images of humans. Similarly to sculptures, creating computer 3D representations require significant skill and a lot of hard work. As with two-dimensional (2D) images, technology can and already has made creating 3D images (or surface meshes) easier.

Depth cameras---cameras that can measure depth in addition to saving a planar image---have recently become cheap and widely available. Normal cameras can automate the creation of 2D images, and depth cameras are on their way to automating the creation of 3D images.

While the depth camera technology is old, there were no consumer devices before Microsoft introduced Kinect for Xbox 360. Kinect contains a normal camera for visible light and also includes an infrared (IR) camera and a laser emitter for depth measurements. Devices that measure color and depth are called RGB-D sensors---RGB for red, green and blue, and D for depth.

The technology used in Kinect is based on measuring the displacement of a known pattern, which can be used to calculate a 3D geometry. This method, called triangulation, is not the only possible one. Another variety of methods called time-of-flight is used in many industrial depth sensors, and is based on measuring the time taken for a laser ray to travel.

Originally, Kinect could only be used with Xbox 360. However, hobbyists quickly began projects to connect the Kinect to a PC. The API was soon reverse engineered. Later on Microsoft released an SDK to the public, allowing people to experiment with Kinect on a Windows PC.

Afterwards, other consumer sensors have been released. Asus Xtion Pro and Asus Xtion Pro Live use the same PrimeSense PS1080 sensor hardware as Kinect does. Alternatives do exist, too; the Creative Interactive Gesture Camera and SoftKinetic DepthSense are time-of-flight based depth cameras at consumer prices.

Cheap depth cameras can already be used for 3D scanning static objects at home. However, to date there has been no software available for scanning humans moving freely. There's no fundamental reason why this can't be done. The possibility to create a 3D model of oneself is intriguing. This would have an enormous amount of applications, for example \fixme{in gaming, movie effects, fitting clothes, ...}.

\fixme{TODO: make explicit constraints, i.e. formulate the problem accurately. Present a research question.}

The possibilities granted by a technology to efficiently scan moving people are vast, but so are the technological details that could be accounted for. In an effort to pick a reasonable research subject, some seemingly arbitrary constraints must be chosen.

Firstly, we choose to pursue an actual working implementation (no matter how crude) instead of only a description of methods. We try to reach an implementation that works on a high-end PC using a single Microsoft Kinect, and therefore could easily be utilized by most \fixme{hobbyists}. We arbitrarily choose to develop on Linux, but strive to make the implementation multiplatform.

Finally, we require that the implementation works in real time. This is because methods for offline modeling have already been presented (though there's no implementation available).

The previously mentioned problem and constraints are formulated as the following research question:

\fixme{Is it possible to generate a 3D model of the user in real time, using Microsoft Kinect (or equivalent sensor) and a consumer PC?} \fixme{TODO: mention motion; (surely this is possible, but how well?)}
