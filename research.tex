\chapter{Approach}

\section{Getting RGB-D data}

% TODO: explain RGB-D
% TODO: avoid using an abbreviation altogether?

The OpenNI library \citep{OpenNI}
was used for data acquisition. The actual sensor used was Microsoft Kinect.

\section{Person recognition and skeleton generation}

Before any attempts at human body reconstruction can be made, the body needs to be detected from the RGB-D image. 

The NITE middleware for OpenNI \citep{NITE}
includes functions for person detection. Moreover, NITE has the capability to generate a skeleton representation of human users.

As the skeleton representation is available, it was used to aid in recosntructing the body model.

\section{Body part segmentation}

Using the skeleton data, the RGB-D point cloud was segmented according to body parts.

The body parts used were chosen according to what the NITE skeleton allowed. Thus the following segments were used:

\fixme{TODO: what do we really have? what are the right English terms?
legs, feet, forearms, shoulders, head, torso, hip}

\section{Point cloud alignment}

The naïve approach to modeling uses data only from a single frame. However, this is unsatisfactory as in practice only less than a half (e.g. the front part) of a human can be seen at once.

To allow for combining data from multiple frames, the observations (point clouds) need to be aligned one way or another.

\section{Mesh generation}

\fixme{Alternative approaches, including the following}

\subsection{Voxel grid}

One possible approach is to use a voxel grid not unlike the one Kinect Fusion \citep{} uses. This allows generating meshes representing arbitrary shapes.

\subsection{Parametric surfaces}

Another approach is to use the information readily available about the body parts being modeled. Instead of allowing arbitrary shapes, the space of possible shapes can be limited to what the body parts tend to look like.

\section{Texturing}

\fixme{Write something here once I've done something...}
