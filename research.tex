\chapter{Approach}
% TODO: remember to justify the choices made
% also compare different possibilities

\section{Obtaining point cloud data}

% TODO: explain RGB-D
% TODO: avoid using an abbreviation altogether?

The OpenNI library \citep{OpenNI}
was used for data acquisition. The actual sensor used was Microsoft Kinect.

\section{Body part segmentation}

Before any attempts at human body reconstruction can be made, the body needs to be detected from the RGB-D image. % mention 'segmentation'

The NITE middleware for OpenNI \citep{NITE}
includes functions for person detection. Moreover, NITE has the capability to generate a skeleton representation of human users.

As the skeleton representation is available, we used it to aid in reconstructing the body model.

Using the skeleton data, we segment the point cloud according to body parts.

\fixme{The body parts used were chosen according to what the NITE skeleton allowed.} Thus the following segments were used:

\fixme{TODO: what do we really have? what are the right English terms?
legs, feet, forearms, shoulders, head, torso, hip}

\section{Point cloud alignment}

The naïve approach to modeling uses data only from a single frame. However, this is unsatisfactory as in practice only less than a half (e.g. the front part) of a human can be seen at once. Another consideration is that the data tends to be noisy and inaccurate. Accumulating data over time makes it possible to remove some of the noise and improve accuracy.

To allow for combining data from multiple frames, the observations (point clouds) need to be aligned one way or another. Methods introduced in \autoref{literature.alignment} were available, each with their own trade-offs.

\fixme{write more about this when the best approach is actually chosen}

\section{Mesh generation}

\fixme{Alternative approaches, including the following}

\subsection{Voxel grid}

One possible approach is to use a voxel grid similar to the one Kinect Fusion \citep{} uses. This allows generating meshes representing arbitrary shapes.

To evaluate the approach, a prototype was built on top of Kinfu\footnote{Kinfu is an open source implementation of Kinect Fusion \citep{newcombe2011kinectfusion}. It is included in the Point Cloud Library \citep{PCL}.}. Since the point clouds were already segmented by body part, it was possible to create recordings that only include a single body part. These recordings could then be played back and used as input for Kinfu.

The working hypothesis was that each body part could be treated as a static object, and that Kinfu should do quite well at modeling them. \fixme{Notably there's little difference between the camera moving} (as is the case in Kinect Fusion) and the object moving. If the background is filtered out, the result is similar. This made for the case that running the Kinect Fusion algorithm in parallel to each body part could give reasonably good results.

In practice, Kinfu doesn't work well with isolated limb data. This has multiple causes. \fixme{1) For full-body scanning the whole user must obviously fit in the picture. At the VGA resolution that Kinect uses, this leaves few pixels per limb. 2) The depth resolution of Kinect is about 1--2 centimeters at a distance suitable for full-body scanning. 3) ICP only uses surface features for alignment, not color data. As the individual body parts tend to be quite smooth, this is a problem.}

\subsection{Parametric surfaces}

Another approach is to use the information readily available about the body parts being modeled. Instead of allowing arbitrary shapes, the space of possible shapes can be limited to what the body parts tend to look like.

As the very first prototype, a simple cylinder approximation was used. A cylinder was fitted to the point cloud of a body part, and colored according to its average color. This was good for testing the segmentation and getting a general idea of how accurate the skeleton is. Some corner cases were also found using this prototype. \fixme{elaborate...}

Obviously, a more human-like model was needed. The "cylinder man" could be usable as an avatar, and is certainly recognizable as human-like. But certainly its shape was nowhere near a real human body. Different simple shapes such as ellipsoids were considered. This would still have left the problem of connecting the body parts. Some kind of a meta-blob solution might have been feasible, but in the end this approach started to seem quite cumbersome. And still the accuracy would leave much to be hoped for.

The best parametric model for the purpose would then be something that already assumes the generic human shape, while allowing \fixme{exact} modifications to single body parts. The SCAPE model \citep{anguelov2005scape} used by \citet{weiss2011home} fits the description, but is not suitable for real-time evaluation. A more suitable approach is taken by MakeHuman \citep{makehuman}, which uses a base mesh and has defined parameters that can be used to reshape different parts of the mesh.

MakeHuman is still in development and at the time of writing was undergoing some large changes. It's not designed to be used as a library, either. Interfacing with other software thus seemed nontrivial. After some time tinkering with the MakeHuman implementation, it was decided that the data is important while the software can be replaced. The most important features weren't very difficult to implement.

\fixme{Where should the implementation details of MakeHuman be described?}

\section{Texturing}

\fixme{Write something here once I've done something...}


