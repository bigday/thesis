% thesis using aalto-thesis.sty
% http://cse.aalto.fi/studies/instructions/tips-for-masters-thesis-workers/
\documentclass[12pt,a4paper,oneside]{report}
\usepackage[utf8]{inputenc}
\usepackage[OT1]{fontenc}
\usepackage[english]{babel}

\usepackage[square]{natbib}

% The aalto-thesis package provides typesetting instructions for the
% standard master's thesis parts (abstracts, front page, and so on)
% Load this package second-to-last, just before the hyperref package.
% Options that you can use: 
%   mydraft - renders the thesis in draft mode. 
%             Do not use for the final version. 
%   doublenumbering - [optional] number the first pages of the thesis
%                     with roman numerals (i, ii, iii, ...); and start
%                     arabic numbering (1, 2, 3, ...) only on the 
%                     first page of the first chapter
%   twoinstructors  - changes the title of instructors to plural form
%   twosupervisors  - changes the title of supervisors to plural form
\usepackage[mydraft,doublenumbering]{aalto-thesis}


% Hyperref
% ------------------------------------------------------------------
% Hyperref creates links from URLs, for references, and creates a
% TOC in the PDF file.
% This package must be the last one you include, because it has
% compatibility issues with many other packages and it fixes
% those issues when it is loaded.   
\RequirePackage[pdftex]{hyperref}
% Setup hyperref so that links are clickable but do not look 
% different
\hypersetup{colorlinks=false,raiselinks=false,breaklinks=true}
\hypersetup{pdfborder={0 0 0}}
\hypersetup{bookmarksnumbered=true}
% The following line suggests the PDF reader that it should show the 
% first level of bookmarks opened in the hierarchical bookmark view. 
\hypersetup{bookmarksopen=true,bookmarksopenlevel=1}
% Hyperref can also set up the PDF metadata fields. These are
% set a bit later on, after the thesis setup.   

% Thesis setup
% ==================================================================
% Change these to fit your own thesis.
% \COMMAND always refers to the English version;
% \FCOMMAND refers to the Finnish version; and
% \SCOMMAND refers to the Swedish version.
% You may comment/remove those language variants that you do not use
% (but then you must not include the abstracts for that language)
% ------------------------------------------------------------------
\newcommand{\TITLE}{Software Processes for Dummies:}
\newcommand{\SUBTITLE}{Re-inventing the Wheel}
\newcommand{\DATE}{June 18, 2011}

\newcommand{\SUPERVISOR}{Professor Antti Ylä-Jääski}
\newcommand{\INSTRUCTOR}{Olli Ohjaaja M.Sc. (Tech.)}


% Other stuff
% ------------------------------------------------------------------
\newcommand{\PROFESSORSHIP}{Data Communication Software}
% Professorship code is the same in all languages
\newcommand{\PROFCODE}{T-110}
\newcommand{\KEYWORDS}{ocean, sea, marine, ocean mammal, marine mammal, whales,
cetaceans, dolphins, porpoises}
\newcommand{\LANGUAGE}{English}

% Author is the same for all languages
\newcommand{\AUTHOR}{Hannu Hartikainen}


% Currently the English versions are used for the PDF file metadata
% Set the PDF title
\hypersetup{pdftitle={\TITLE\ \SUBTITLE}}
% Set the PDF author
\hypersetup{pdfauthor={\AUTHOR}}
% Set the PDF keywords
\hypersetup{pdfkeywords={\KEYWORDS}}
% Set the PDF subject
\hypersetup{pdfsubject={Master's Thesis}}


% Layout settings
% ------------------------------------------------------------------

% Use this to control how much space there is between each line of text.
% 1 is normal (no extra space), 1.3 is about one-half more space, and
% 1.6 is about double line spacing.  
\linespread{1.3}

% Bibliography style
\bibliographystyle{plainnat}


% The preamble ends here, and the document begins. 
% Place all formatting commands and such before this line.
% ------------------------------------------------------------------
\begin{document}
% This command adds a PDF bookmark to the cover page. You may leave
% it out if you don't like it...
\pdfbookmark[0]{Cover page}{bookmark.0.cover}
% This command is defined in aalto-thesis.sty. It controls the page 
% numbering based on whether the doublenumbering option is specified
\startcoverpage

% Cover page
% ------------------------------------------------------------------
% Options: finnish, english, and swedish
% These control in which language the cover-page information is shown
\coverpage{english}


% Abstracts
% ------------------------------------------------------------------
% Include an abstract in the language that the thesis is written in,
% and if your native language is Finnish or Swedish, one in that language.

% Abstract in English
% ------------------------------------------------------------------
\thesisabstract{english}{
\fixme{Abstract text goes here}
}


% Acknowledgements
% ------------------------------------------------------------------
\selectlanguage{english}
\chapter*{Acknowledgements}

I wish to thank all students who use \LaTeX\ for formatting their theses,
because theses formatted with \LaTeX\ are just so nice.

Thank you, and keep up the good work!
\vskip 10mm

\noindent Espoo, \DATE
\vskip 5mm
\noindent\AUTHOR


% Acronyms
% ------------------------------------------------------------------
% Use \cleardoublepage so that IF two-sided printing is used 
% (which is not often for masters theses), then the pages will still
% start correctly on the right-hand side.
%\cleardoublepage
%\cleardoublepage
\phantomsection
\addcontentsline{toc}{chapter}{Acronyms}
\chapter*{Acronyms}

% The longtable environment should break the table properly to multiple pages, 
% if needed

\noindent
\begin{longtable}{@{}p{0.25\textwidth}p{0.7\textwidth}@{}}
    3D; 3-D & 3-Dimensional \\
        API & Application Programming Interface \\
     CAESAR & Civilian American and European Surface Anthropometry Resource \\
        CLI & Common Language Infrastructure \\
        CPU & Central Processing Unit \\
         GB & Gigabyte(s) \\
        GPL & (GNU) General Public License \\
      GPGPU & General Purpose computing on Graphics Processing Units \\
        GPU & Graphics Processing Unit \\
        GUI & Graphical User Interface \\
         EM & Expectation-Maximization \\
     EM-ICP & Expectation-Maximization ICP \\
       EULA & End-User License Agreement \\
       FAST & Features from Accelerated Segment Test \\
      FOVIS & Fast Odometry from VISion \\
        fps & Frames per second \\
        ICP & Iterative Closest Point \\
         IR & Infrared \\
       LGPL & (GNU) Lesser General Public License \\
        MAV & Micro Air Vehicle \\
         MB & Megabyte(s) \\
       NITE & Natural Interaction Technology for End-user \\
     OpenNI & Open Natural Interaction \\
        PCA & Principal Component Analysis \\
        PCL & Point Cloud Library \\
        RAM & Random Access Memory \\
        RGB & Red, Green, Blue \\
RGB-D; RGBD & Red, Green, Blue, Depth \\
      SCAPE & Shape Completion and Animation of PEople \\
        SDF & Signed Distance Function \\
        SDK & Software Development Kit \\
       SLAM & Simultaneous localization and mapping \\
        SSD & Solid-State Drive \\
%   TPS-RPM & \\
       TSDF & Truncated Signed Distance Function \\
\end{longtable}



% Table of contents
% ------------------------------------------------------------------
\cleardoublepage
% This command adds a PDF bookmark that links to the contents.
% You can use \addcontentsline{} as well, but that also adds contents
% entry to the table of contents, which is kind of redundant.
% The text ``Contents'' is shown in the PDF bookmark. 
\pdfbookmark[0]{Contents}{bookmark.0.contents}
\tableofcontents

% The following label is used for counting the prelude pages
\label{pages-prelude}
\cleardoublepage


%%%%%%%%%%%%%%%%% The main content starts here %%%%%%%%%%%%%%%%%%%%%
% ------------------------------------------------------------------
% This command is defined in aalto-thesis.sty. It controls the page 
% numbering based on whether the doublenumbering option is specified
\startfirstchapter

% Add headings to pages (the chapter title is shown)
\pagestyle{headings}

% The contents of the thesis are separated to their own files.
% Edit the content in these files, rename them as necessary.
% ------------------------------------------------------------------

%\chapter{Introduction}

\fixme{Background:} Depth cameras have recently become cheap and widely available. While the technology is old, there were no consumer devices before Microsoft introduced their Kinect for Xbox 360. Kinect contains a normal camera for visible light, \fixme{and} also includes an IR camera and a laser emitter for depth measurements. \fixme{TODO: write about technologies -- time-of-flight, triangulation etc. (perhaps in literature section?)}

Originally, Kinect was only supported for Xbox 360. However, hobbyists quickly began projects to connect the Kinect to a PC. The API was soon reverse engineered. Later on Microsoft released an SDK to the public, allowing people to experiment with Kinect on a Windows PC.

Afterwards, other consumer sensors were released. \fixme{Asus Xtion Pro/Live, what else}

\fixme{TODO: explain the problem we're trying to solve here}
Cheap depth cameras can already be used for 3D scanning static objects at home. However, to date there has been no software available for scanning humans moving freely. There's no fundamental reason why this can't be done. The possibility to create a 3D model of oneself is intriguing. This would have an enormous amount of applications, for example \fixme{in gaming, movie effects, fitting clothes, ...}.

\fixme{TODO: make explicit constraints, i.e. formulate the problem accurately. Present a research question.}

The possibilities granted by a technology to efficiently scan moving people are vast, but so are the technological details that could be accounted for. In an effort to pick a reasonable research subject, some seemingly arbitrary constraints must be chosen.

Firstly, we choose to pursue an actual working implementation (no matter how crude) instead of only a description of methods. We try to reach an implementation that works on a high-end PC using a single Microsoft Kinect, and therefore could easily be utilized by most \fixme{hobbyists}. We arbitrarily choose to develop on Linux, but strive to make the implementation multiplatform.

Finally, we require that the implementation works in real time. This is because methods for offline modeling have already been presented (though there's no implementation available).

The previously mentioned problem and constraints are formulated as the following research question:

\fixme{Is it possible to generate a 3D model of the user in real time, using Microsoft Kinect (or equivalent sensor) and a consumer PC?}

% ...


% Load the bibliographic references
% ------------------------------------------------------------------
% You can use several .bib files:
% \bibliography{thesis_sources,ietf_sources}
\bibliography{sources}


% Appendices go here
% ------------------------------------------------------------------
% If you do not have appendices, comment out the following lines
%\appendix
%\input{appendices.tex}

% End of document!
% ------------------------------------------------------------------
% The LastPage package automatically places a label on the last page.
% That works better than placing a label here manually, because the
% label might not go to the actual last page, if LaTeX needs to place
% floats (that is, figures, tables, and such) to the end of the 
% document.
\end{document}
